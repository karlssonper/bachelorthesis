\subsection{Background}

There have mainly been two different traditional ways of trying to solve the fluid equations, the Lagrangian viewpoint and the Eulerian viewpoint. The Lagrangian approach treats continuum like a particle system where each point in the fluid is labeled as a particle with a position and velocity. The Smoothed Particle Hydrodynamics[TODO REF], SPH, was introduced to computational fluid dynamics and animation by Desbrun and Cani[REF TODO]. The big advantage of SPH is that the advection step does not suffer from any numerical smoothing and act directly on the particles. It gets trickier to approximate spatial derivates since it completely depends on where and how many particles there are nearby. This can lead to instability in the simulation. Eulerian methods are grid based and instead of tracking a part of the fluid as it moves, they focus on tracking a fixed point in space and check how the fluid quantites at that point are changing over time. Foster and Bla[TODO] introduced the first 3D grid based water simulation. One problem in the beginning of grid based solutions was that it tended to blow up eventually and it was not stable for long simulations. J.Stam[TODO] added a semi-langrangian advection to the grid based methods and made it unconditionally stable which allowed larger timesteps than before. To track surfaces, signed distance fields and level sets were used. Fedkiw and Foster[TODO] introduced a way to combine particles and level set for better surface tracking and therefore have less volume loss in the simulation. Unlike the Lagrangian approach, approximating a spatial derivate is a lot easier on a grid where one can just look at the difference between neighboring cells. Grid methods have many steps that requires interpolation, which tend to smooth out interesting high frequency motion in the fluid.
\newline
\newline
Both the Lagrangian and Eulerian have their advantages and disadvantges. Zhou[TODO] came up with a way of combining them to have a stable pressure solving step on a grid and a Lagrangian advection step. This method is usually called FLIP. Zhou and Bridson used a preconditioned conjugate gradient solver for solving the pressure equations. In this report we are going to use the Multigrid approach instead, inspired by Mueller[12].
