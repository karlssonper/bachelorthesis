In general, Computer graphics tries to come up with methods to represent what we see and interpret with our eyes. The main target has always been to have a image in the end that looks reasonable. There are two restrictions, time and accuracy. We don't have unlimited time to produce each image in most cases, especially in animation where there are 24 images per second that has to constructed. The other limitation is accuracy. Some problems we don't know how to model and some problems we don't know how to solve. Simplifications are accepted as long as the end result looks covincing. 
\newline
\newline
In the field of animation is we care about how objects move in space and how they deform over time. Many problems can be solved by simply moving objects in space manually, often refered to as hand animating. A common technique for animating characters in the visual effects and video games industry is to only animated a simplicated skeleton and then weight geometry to different parts of the skeleton. This is fine for a characters main movement. However, for more complex things on characters, like hair and cloth, simple explicit approaches don't work anymore. There are too many hair strands or wrinkles on the cloth to control them all manually. Even if we would attempt to, it is very likely that the motion would look unreal. For more complex phenonoma we need to simulate the objects and geometry for a convincing motion.
\newline
\newline
This report is focusing on the simualation of the interesting water effects. Liquid fluids have, unlike their gas state, a surface defining their volume. The surface has a lot of degrees of freedom and can take any shape. To make it even more complex, it also changes topology over time which makes it tricky for implementations to track the surface. The other challenging part of fluid simulations is to define how the velocity in the fluid is changing over time where each little subregion of the fluid can have a velocity completely different from its neighbors. 
