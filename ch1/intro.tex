In general, computer graphics tries to come up with methods to represent what we see and interpret with our eyes, where the main goal has always been to produce an image in the end that looks reasonable. There are two restrictions, time and accuracy. In most cases, we do not have unlimited time to produce each image, especially not in animation where there are 24 images per second that have to be constructed. The other limitation is accuracy. Some problems we do not know how to model and some problems we do not know how to solve precisely. Simplifications are accepted as long as the end result looks convincing. 
\newline
\newline
In the field of animation, we care about how objects move in space and how they deform over time. Many problems can be solved by simply moving objects in space manually, often referred to as hand-animating. A common technique for animating characters in the visual effects and video games industry is to only animated a simplified skeleton and then weight geometry to different parts of the skeleton. This is fine for a characters main movement. However, for more complex things on characters, like hair and cloth, simple hand-animated approaches do not work anymore. There are too many hair strands or wrinkles on the cloth to control them all manually. Even if we would attempt to, it is very likely that the motion would look unreal. For more complex phenomena we need to simulate the objects and their geometry for a convincing motion.
\newline
\newline
This report focus on simulation of interesting water effects. Liquid fluids have, unlike their gas state, a surface defining their volume. The surface has a lot of degrees of freedom and can take any shape. To make it even more complex, it also changes topology over time which makes it hard to write surface tracking implementations. The other challenging part of fluid simulation is to define how the velocity in the fluid is changing over time where each little part of the fluid can have a velocity completely different from its neighbors. 
