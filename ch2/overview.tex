\subsection{Outline of the Algorithm}

This section will briefly outline the important steps of the FLIP algorithm that are covered later on in the report. The key to the hybrid FLIP method is that we store both particles and a grid. Each particle stores a position in space, $\vec{x}_p$ and a velocity $\vec{u}_p$. The particles and the grid are going to comminucate and transfer properties to each other. The particles are integrated over time to update the positions. The particle velocities are later transferred to the grid and it is up to the grid to make the velocity field divergence-free and therefore conserving the volume of the water. The input to the algorithm is the initial geometry shape of the fluid and the collidable solid boundaries, i.e. walls or rigid objects that the water should collide against. We will refer to the solid boundaries as just solids in this report. We will use a level set to track the surface of the solids.

\begin{enumerate}
\item Create particles from initial shape - \emph{Ch 3.1}
\item Transfer particle velocities to grid - \emph{Ch 3.2}
\item Apply external forces to grid
\item Mark cells fluid - \emph{Ch 4.1}
\item Create level set - \emph{Ch 4.1}
\item Reinitialize level set - \emph{Ch 4.2}
\item Extrapolate velocities - \emph{Ch 4.3}
\item Solve pressure equations- \emph{Ch 5}
\item Transfer grid velocities to particles - \emph{Ch 3.3}
\item Advect particles - \emph{Ch 3.4}
\item Repeat step 2-10 until simulation done
\end{enumerate}

The only item not covered later on in this report is \emph{Apply external forces to grid}. This step is straightforward and can be done with a simple Euler step:

\begin{equation}
\vec{u}^{ext} = \vec{u} + \Delta t \cdot \vec{g}
\end{equation}
